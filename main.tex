\documentclass{article}

% Language setting
% Replace `english' with e.g. `spanish' to change the document language
\usepackage[english]{babel}

% Set page size and margins
% Replace `letterpaper' with `a4paper' for UK/EU standard size
\usepackage[letterpaper,top=2cm,bottom=2cm,left=3cm,right=3cm,marginparwidth=1.75cm]{geometry}

% Useful packages
\usepackage{amsmath}
\usepackage{graphicx}
\usepackage[colorlinks=true, allcolors=blue]{hyperref}

\title{Food Industry}
\author
{ 
    Juan Martin Parra Garcia \\
    Leonel Martinez Huitron \\
    Fabian Alonso Cobian Garcia
}

\begin{document}
\maketitle

\begin{abstract}

\end{abstract}

\section{Overall Objective}

Our objective is to handle the distribution and quality assurance (QA) in the food industry, registering when a batch of food arrives and leaves. We ensure that our food complies with required regulations and standards and delivers on time. 
Currently, we are not altering our products in any way, so they remain in their original state.


\section{General Structure}


Currently, our structure is still in development, so it is susceptible to changes in the future. However, at present, there are four categories in our structure:
Quality, registering, distribution, administrative.
These classifications in our structure are linked to each other in some way or another.
Quality:
In order for food to be distributed, it needs to be ISO compliant. This category focuses on the regulations necessary to achieve compliance. For example, it involves inspecting batches of food by QA. 
\\
\textbf{Registering:}

For a successful workflow and to minimize errors, it's important to document available information.
This includes maintaining records of inspection results, client requirements, and information provided by our distributors.
\\
\textbf{distribution:}

An organized distribution it´s a good way to keep a good relationship with our clients and distributors, but an distribution that keeps a records of whats happening its a must.
\\
\textbf{Administrative:}

Efficiently handling client needs requires a formalized structure for recording both old and new transactions. The administrative sector is not only focused on external interactions with clients and suppliers, but also on managing internal operations. This includes overseeing employee-related matters and ensuring an effective internal work structure.

\begin{figure}
\centering
\includegraphics[width=0.25\linewidth]{industryfd.jpg}
\end{figure}


\section{Data Management in the Food Industry}
In the food industry, there is a need for regulations and standards to meet certain deadlines because, being food, it expires after a certain period of time. Therefore, record-keeping facilitates and helps maintain a workflow in line with demand. 
With so many requirements for regulations, records, and distribution, it is also necessary to establish a database capable of managing and administering this information and converting it into useful data when needed.

\section{Consumer Data Analysis and Trends}
\begin{enumerate}
    \item Explore how databases can be used to collect and analyze data on consumer trends in the food industry.
    \item Analyze how data collected from sources such as sales transactions, consumer surveys, and social media analytics can be used to identify purchasing patterns, product preferences, and consumer behaviors.
    
    \item Discuss the importance of understanding consumer trends for marketing strategies, product development, and business decision-making in the food industry.
    
    \item Examine data analysis techniques and tools, such as regression analysis, clustering, and time series analysis, that can be applied to food industry data to identify insights and business opportunities.
    
    \item Consider the challenges associated with analyzing large volumes of data in the food industry, such as data quality, consumer privacy, and accurate interpretation of analysis results.
    
\end{enumerate}

\section{Supply Chain Optimization and Inventory Management}
\begin{enumerate}
    \item Explore how databases can be utilized to optimize supply chain management in the food industry, from production to distribution and point of sale.
    
    \item Analyze how data collected from different stages of the supply chain can be used to predict demand, optimize inventory levels, and reduce operational costs.
    
    \item Discuss the importance of integrating inventory management systems based on databases with other technologies, such as automation and advanced logistics, to enhance efficiency and accuracy in inventory management.
    
    \item Examine common challenges associated with inventory management in the food industry, such as demand seasonality, variability in product quality, and managing perishable products.
    
    \item Consider strategies and best practices for implementing inventory management systems based on databases in the food industry, including staff training, selecting appropriate technology, and integration with other business systems.

\end{enumerate}

\end{document}